% !TEX program = xelatex
% !BIB program = bibtex

\documentclass[UTF8,cs4size]{ctexart}

% layout
\usepackage[left=3cm,right=3cm]{geometry}
\linespread{1}
\ctexset{
  section = {
    name = \S
  },
  subsection/name = \S,
  subsubsection/name = \S
}
% \makeatletter
% \def\@seccntformat#1{%
%   \expandafter\ifx\csname c@#1\endcsname\c@section
%   Section \thesection:
%   \else
%   \csname the#1\endcsname\quad
%   \fi}
% \makeatother
 
% page headings
\usepackage{fancyhdr}
\setlength{\headheight}{15.2pt}
\pagestyle{fancy}
\lhead{\leftmark}
\rhead{M201873026 刘一龙}
\cfoot{\thepage}
% \makeatletter
% \let\headauthor\@author
% \makeatother

% url/ref
\usepackage{hyperref}
\hypersetup{
  colorlinks,
  citecolor=black,
  filecolor=black,
  linkcolor=black,
  urlcolor=black,
  pdfauthor={刘一龙},
  pdftitle={人工智能结课报告}
}

% vertical centering title page
\usepackage{titling}
\renewcommand\maketitlehooka{\null\mbox{}\vfill}
\renewcommand\maketitlehookd{\vfill\null}

% table of contents
\usepackage{tocloft}
\renewcommand\cftsecfont{\normalfont}
\renewcommand\cftsecpagefont{\normalfont}
\renewcommand{\cftsecleader}{\cftdotfill{\cftsecdotsep}}
\renewcommand\cftsecdotsep{\cftdot}
\renewcommand\cftsubsecdotsep{\cftdot}
\renewcommand{\contentsname}{\hfill\bfseries\Large 目录\hfill}   
\setlength{\cftbeforesecskip}{10pt}

% figures
\usepackage{graphicx}
\graphicspath{figures/}
% \newcommand\figureht{\dimexpr
%   \textheight-3\baselineskip-\parskip-.2em-
%   \abovecaptionskip-\belowcaptionskip\relax}

% tables
\usepackage{caption} 
\captionsetup[table]{skip=10pt}

% math, algorithms, code
\usepackage{amsmath,amssymb,url}
\usepackage{algorithm, algorithmic}
\usepackage{listings}

\lstset{
   extendedchars=true,
   basicstyle=\footnotesize\ttfamily,
   showstringspaces=false,
   showspaces=false,
   numbers=left,
   numberstyle=\footnotesize,
   numbersep=9pt,
   tabsize=2,
   breaklines=true,
   showtabs=false,
   captionpos=b
}

% bibliography
\usepackage[super,square,comma,sort]{natbib} % for \citet and \citep
\renewcommand{\refname}{\S 参考文献}
% \begin{filecontents}{report.bib}
% \end{filecontents} 

% appendix
\usepackage{appendix}

\title{人工智能结课报告\\ \bigskip \textbf{GBA: A Simple Gobang AI Based on Monte-Carlo Tree Search Algorithm}}
\author{计算机科学与技术学院\\ 硕1801\\ M201873026\\ 刘一龙}
\date{\today}

\begin{document}

\pagenumbering{gobble} % no page number

\maketitle
\newpage

% empty page
\null\thispagestyle{empty}
\newpage

\pagenumbering{roman}

\section*{\S 摘要}\sectionmark{\S 摘要}
\addcontentsline{toc}{section}{\S 摘要}
Monte-Carlo tree search algorithm\cite{wiki:Monte_Carlo_tree_search}\cite{DBLP:conf/ecml/KocsisS06}\cite{DBLP:conf/aiide/ChaslotBSS08}
\newpage

\section*{\S Abstract}\sectionmark{\S Abstract}
Monte-Carlo tree search algorithm\cite{wiki:Monte_Carlo_tree_search}\cite{DBLP:conf/ecml/KocsisS06}\cite{DBLP:conf/aiide/ChaslotBSS08}
\newpage

\pagenumbering{gobble} % no page number

\tableofcontents
\newpage

% empty page
\null\thispagestyle{empty}
\newpage

\pagenumbering{arabic}

\section{引言}
Google Deepmind's AlphaGo AI\cite{DBLP:journals/nature/SilverHMGSDSAPL16}\cite{silver2017mastering}
\newpage

\section{Optimization}
\subsection{Score Board Optimization}
use score board standard to generate legal plays for expansion and simulate phase
\subsection{Result}
smarter
\newpage

\bibliographystyle{unsrt}
\bibliography{bibs/report}
\addcontentsline{toc}{section}{\S 参考文献}
% \addcontentsline{toc}{section}{\protect\numberline{}参考文献}
\newpage

\end{document}
